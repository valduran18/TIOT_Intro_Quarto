% Options for packages loaded elsewhere
\PassOptionsToPackage{unicode}{hyperref}
\PassOptionsToPackage{hyphens}{url}
\PassOptionsToPackage{dvipsnames,svgnames,x11names}{xcolor}
%
\documentclass[
  letterpaper,
  DIV=11,
  numbers=noendperiod]{scrartcl}

\usepackage{amsmath,amssymb}
\usepackage{iftex}
\ifPDFTeX
  \usepackage[T1]{fontenc}
  \usepackage[utf8]{inputenc}
  \usepackage{textcomp} % provide euro and other symbols
\else % if luatex or xetex
  \usepackage{unicode-math}
  \defaultfontfeatures{Scale=MatchLowercase}
  \defaultfontfeatures[\rmfamily]{Ligatures=TeX,Scale=1}
\fi
\usepackage{lmodern}
\ifPDFTeX\else  
    % xetex/luatex font selection
\fi
% Use upquote if available, for straight quotes in verbatim environments
\IfFileExists{upquote.sty}{\usepackage{upquote}}{}
\IfFileExists{microtype.sty}{% use microtype if available
  \usepackage[]{microtype}
  \UseMicrotypeSet[protrusion]{basicmath} % disable protrusion for tt fonts
}{}
\makeatletter
\@ifundefined{KOMAClassName}{% if non-KOMA class
  \IfFileExists{parskip.sty}{%
    \usepackage{parskip}
  }{% else
    \setlength{\parindent}{0pt}
    \setlength{\parskip}{6pt plus 2pt minus 1pt}}
}{% if KOMA class
  \KOMAoptions{parskip=half}}
\makeatother
\usepackage{xcolor}
\setlength{\emergencystretch}{3em} % prevent overfull lines
\setcounter{secnumdepth}{-\maxdimen} % remove section numbering
% Make \paragraph and \subparagraph free-standing
\ifx\paragraph\undefined\else
  \let\oldparagraph\paragraph
  \renewcommand{\paragraph}[1]{\oldparagraph{#1}\mbox{}}
\fi
\ifx\subparagraph\undefined\else
  \let\oldsubparagraph\subparagraph
  \renewcommand{\subparagraph}[1]{\oldsubparagraph{#1}\mbox{}}
\fi

\usepackage{color}
\usepackage{fancyvrb}
\newcommand{\VerbBar}{|}
\newcommand{\VERB}{\Verb[commandchars=\\\{\}]}
\DefineVerbatimEnvironment{Highlighting}{Verbatim}{commandchars=\\\{\}}
% Add ',fontsize=\small' for more characters per line
\usepackage{framed}
\definecolor{shadecolor}{RGB}{241,243,245}
\newenvironment{Shaded}{\begin{snugshade}}{\end{snugshade}}
\newcommand{\AlertTok}[1]{\textcolor[rgb]{0.68,0.00,0.00}{#1}}
\newcommand{\AnnotationTok}[1]{\textcolor[rgb]{0.37,0.37,0.37}{#1}}
\newcommand{\AttributeTok}[1]{\textcolor[rgb]{0.40,0.45,0.13}{#1}}
\newcommand{\BaseNTok}[1]{\textcolor[rgb]{0.68,0.00,0.00}{#1}}
\newcommand{\BuiltInTok}[1]{\textcolor[rgb]{0.00,0.23,0.31}{#1}}
\newcommand{\CharTok}[1]{\textcolor[rgb]{0.13,0.47,0.30}{#1}}
\newcommand{\CommentTok}[1]{\textcolor[rgb]{0.37,0.37,0.37}{#1}}
\newcommand{\CommentVarTok}[1]{\textcolor[rgb]{0.37,0.37,0.37}{\textit{#1}}}
\newcommand{\ConstantTok}[1]{\textcolor[rgb]{0.56,0.35,0.01}{#1}}
\newcommand{\ControlFlowTok}[1]{\textcolor[rgb]{0.00,0.23,0.31}{#1}}
\newcommand{\DataTypeTok}[1]{\textcolor[rgb]{0.68,0.00,0.00}{#1}}
\newcommand{\DecValTok}[1]{\textcolor[rgb]{0.68,0.00,0.00}{#1}}
\newcommand{\DocumentationTok}[1]{\textcolor[rgb]{0.37,0.37,0.37}{\textit{#1}}}
\newcommand{\ErrorTok}[1]{\textcolor[rgb]{0.68,0.00,0.00}{#1}}
\newcommand{\ExtensionTok}[1]{\textcolor[rgb]{0.00,0.23,0.31}{#1}}
\newcommand{\FloatTok}[1]{\textcolor[rgb]{0.68,0.00,0.00}{#1}}
\newcommand{\FunctionTok}[1]{\textcolor[rgb]{0.28,0.35,0.67}{#1}}
\newcommand{\ImportTok}[1]{\textcolor[rgb]{0.00,0.46,0.62}{#1}}
\newcommand{\InformationTok}[1]{\textcolor[rgb]{0.37,0.37,0.37}{#1}}
\newcommand{\KeywordTok}[1]{\textcolor[rgb]{0.00,0.23,0.31}{#1}}
\newcommand{\NormalTok}[1]{\textcolor[rgb]{0.00,0.23,0.31}{#1}}
\newcommand{\OperatorTok}[1]{\textcolor[rgb]{0.37,0.37,0.37}{#1}}
\newcommand{\OtherTok}[1]{\textcolor[rgb]{0.00,0.23,0.31}{#1}}
\newcommand{\PreprocessorTok}[1]{\textcolor[rgb]{0.68,0.00,0.00}{#1}}
\newcommand{\RegionMarkerTok}[1]{\textcolor[rgb]{0.00,0.23,0.31}{#1}}
\newcommand{\SpecialCharTok}[1]{\textcolor[rgb]{0.37,0.37,0.37}{#1}}
\newcommand{\SpecialStringTok}[1]{\textcolor[rgb]{0.13,0.47,0.30}{#1}}
\newcommand{\StringTok}[1]{\textcolor[rgb]{0.13,0.47,0.30}{#1}}
\newcommand{\VariableTok}[1]{\textcolor[rgb]{0.07,0.07,0.07}{#1}}
\newcommand{\VerbatimStringTok}[1]{\textcolor[rgb]{0.13,0.47,0.30}{#1}}
\newcommand{\WarningTok}[1]{\textcolor[rgb]{0.37,0.37,0.37}{\textit{#1}}}

\providecommand{\tightlist}{%
  \setlength{\itemsep}{0pt}\setlength{\parskip}{0pt}}\usepackage{longtable,booktabs,array}
\usepackage{calc} % for calculating minipage widths
% Correct order of tables after \paragraph or \subparagraph
\usepackage{etoolbox}
\makeatletter
\patchcmd\longtable{\par}{\if@noskipsec\mbox{}\fi\par}{}{}
\makeatother
% Allow footnotes in longtable head/foot
\IfFileExists{footnotehyper.sty}{\usepackage{footnotehyper}}{\usepackage{footnote}}
\makesavenoteenv{longtable}
\usepackage{graphicx}
\makeatletter
\def\maxwidth{\ifdim\Gin@nat@width>\linewidth\linewidth\else\Gin@nat@width\fi}
\def\maxheight{\ifdim\Gin@nat@height>\textheight\textheight\else\Gin@nat@height\fi}
\makeatother
% Scale images if necessary, so that they will not overflow the page
% margins by default, and it is still possible to overwrite the defaults
% using explicit options in \includegraphics[width, height, ...]{}
\setkeys{Gin}{width=\maxwidth,height=\maxheight,keepaspectratio}
% Set default figure placement to htbp
\makeatletter
\def\fps@figure{htbp}
\makeatother

<script src="TIOT_Intro_Quarto_files/libs/htmlwidgets-1.6.3/htmlwidgets.js"></script>
<script src="TIOT_Intro_Quarto_files/libs/plotly-binding-4.10.3/plotly.js"></script>
<script src="TIOT_Intro_Quarto_files/libs/setprototypeof-0.1/setprototypeof.js"></script>
<script src="TIOT_Intro_Quarto_files/libs/typedarray-0.1/typedarray.min.js"></script>
<script src="TIOT_Intro_Quarto_files/libs/jquery-3.5.1/jquery.min.js"></script>
<link href="TIOT_Intro_Quarto_files/libs/crosstalk-1.2.1/css/crosstalk.min.css" rel="stylesheet" />
<script src="TIOT_Intro_Quarto_files/libs/crosstalk-1.2.1/js/crosstalk.min.js"></script>
<link href="TIOT_Intro_Quarto_files/libs/plotly-htmlwidgets-css-2.11.1/plotly-htmlwidgets.css" rel="stylesheet" />
<script src="TIOT_Intro_Quarto_files/libs/plotly-main-2.11.1/plotly-latest.min.js"></script>
\KOMAoption{captions}{tableheading}
\makeatletter
\makeatother
\makeatletter
\makeatother
\makeatletter
\@ifpackageloaded{caption}{}{\usepackage{caption}}
\AtBeginDocument{%
\ifdefined\contentsname
  \renewcommand*\contentsname{Table of contents}
\else
  \newcommand\contentsname{Table of contents}
\fi
\ifdefined\listfigurename
  \renewcommand*\listfigurename{List of Figures}
\else
  \newcommand\listfigurename{List of Figures}
\fi
\ifdefined\listtablename
  \renewcommand*\listtablename{List of Tables}
\else
  \newcommand\listtablename{List of Tables}
\fi
\ifdefined\figurename
  \renewcommand*\figurename{Figure}
\else
  \newcommand\figurename{Figure}
\fi
\ifdefined\tablename
  \renewcommand*\tablename{Table}
\else
  \newcommand\tablename{Table}
\fi
}
\@ifpackageloaded{float}{}{\usepackage{float}}
\floatstyle{ruled}
\@ifundefined{c@chapter}{\newfloat{codelisting}{h}{lop}}{\newfloat{codelisting}{h}{lop}[chapter]}
\floatname{codelisting}{Listing}
\newcommand*\listoflistings{\listof{codelisting}{List of Listings}}
\makeatother
\makeatletter
\@ifpackageloaded{caption}{}{\usepackage{caption}}
\@ifpackageloaded{subcaption}{}{\usepackage{subcaption}}
\makeatother
\makeatletter
\@ifpackageloaded{tcolorbox}{}{\usepackage[skins,breakable]{tcolorbox}}
\makeatother
\makeatletter
\@ifundefined{shadecolor}{\definecolor{shadecolor}{rgb}{.97, .97, .97}}
\makeatother
\makeatletter
\makeatother
\makeatletter
\makeatother
\makeatletter
\@ifpackageloaded{fontawesome5}{}{\usepackage{fontawesome5}}
\makeatother
\ifLuaTeX
  \usepackage{selnolig}  % disable illegal ligatures
\fi
\IfFileExists{bookmark.sty}{\usepackage{bookmark}}{\usepackage{hyperref}}
\IfFileExists{xurl.sty}{\usepackage{xurl}}{} % add URL line breaks if available
\urlstyle{same} % disable monospaced font for URLs
\hypersetup{
  pdftitle={Introduction to Quarto pt.2},
  pdfauthor={Valeria Duran},
  colorlinks=true,
  linkcolor={blue},
  filecolor={Maroon},
  citecolor={Blue},
  urlcolor={Blue},
  pdfcreator={LaTeX via pandoc}}

\title{Introduction to Quarto pt.2}
\author{Valeria Duran}
\date{2023-11-30}

\begin{document}
\maketitle
\ifdefined\Shaded\renewenvironment{Shaded}{\begin{tcolorbox}[boxrule=0pt, breakable, sharp corners, interior hidden, enhanced, frame hidden, borderline west={3pt}{0pt}{shadecolor}]}{\end{tcolorbox}}\fi

\hypertarget{section}{%
\subsection{}\label{section}}

\hypertarget{what-is-quarto}{%
\subsection{What is Quarto?}\label{what-is-quarto}}

\begin{itemize}
\tightlist
\item
  Quarto is an open-source scientific and technical publishing system
  built on Pandoc {(CL interface, not an R package)}
\item
  Next generation version of R Markdown from Posit
\item
  Multi-language: R, Python, Julia, Observable JS
\item
  Authoring tools: RStudio IDE, VSCode, JupyterLab

  \begin{itemize}
  \tightlist
  \item
    Integration with Jupyter enables use of additional languages
  \end{itemize}
\item
  Can render most existing Rmd files
\item
  Has many outputs: publications, books, websites, dashboards
\end{itemize}

\hypertarget{why-is-this-cool}{%
\subsection{Why is this cool?}\label{why-is-this-cool}}

\begin{itemize}
\tightlist
\item
  Git integration

  \begin{itemize}
  \tightlist
  \item
    GitHub Actions allows for Continuous Integration on GitHub
  \end{itemize}
\item
  Interactive/dynamic content
\item
  Single product! Quarto combines many functionalities, leading to fewer
  dependencies to render reports
\item
  Figure and table cross references work in word doc (previously not
  possible with R Markdown)
\item
  Multilingual team\ldots{} {standardize reports}
\item
  Document templates can be created as Quarto extensions
  {\faIcon{star}}\\

  \begin{itemize}
  \tightlist
  \item
    Don't have to build an R package to share the template
  \item
    Can be language-agnostic
  \item
    Works with presentations if we're not using any R code!
  \end{itemize}
\end{itemize}

\hypertarget{interactive-code}{%
\subsection{Interactive Code}\label{interactive-code}}

\subsubsection{R chunk}

\begin{Shaded}
\begin{Highlighting}[]

\InformationTok{\textasciigrave{}\textasciigrave{}\textasciigrave{}\{r\}}
\CommentTok{\#| label: "ggplot{-}example"}
\CommentTok{\#| code{-}line{-}numbers: "5{-}9"}

\FunctionTok{library}\NormalTok{(ggplot2)}
\FunctionTok{library}\NormalTok{(plotly)}
\FunctionTok{library}\NormalTok{(gapminder)}

\NormalTok{p }\OtherTok{\textless{}{-}}\NormalTok{ gapminder }\SpecialCharTok{\%\textgreater{}\%}
  \FunctionTok{filter}\NormalTok{(year}\SpecialCharTok{==}\DecValTok{1977}\NormalTok{) }\SpecialCharTok{\%\textgreater{}\%}
  \FunctionTok{ggplot}\NormalTok{( }\FunctionTok{aes}\NormalTok{(gdpPercap, lifeExp, }\AttributeTok{size =}\NormalTok{ pop, }\AttributeTok{color=}\NormalTok{continent)) }\SpecialCharTok{+}
  \FunctionTok{geom\_point}\NormalTok{() }\SpecialCharTok{+}
  \FunctionTok{theme\_bw}\NormalTok{()}

\FunctionTok{ggplotly}\NormalTok{(p)}


\InformationTok{\textasciigrave{}\textasciigrave{}\textasciigrave{}}
\end{Highlighting}
\end{Shaded}

\subsubsection{Highlighted Code}

\begin{Shaded}
\begin{Highlighting}[numbers=left,,]
\FunctionTok{library}\NormalTok{(ggplot2)}
\FunctionTok{library}\NormalTok{(plotly)}
\FunctionTok{library}\NormalTok{(gapminder)}

\NormalTok{p }\OtherTok{\textless{}{-}}\NormalTok{ gapminder }\SpecialCharTok{\%\textgreater{}\%}
  \FunctionTok{filter}\NormalTok{(year}\SpecialCharTok{==}\DecValTok{1977}\NormalTok{) }\SpecialCharTok{\%\textgreater{}\%}
  \FunctionTok{ggplot}\NormalTok{( }\FunctionTok{aes}\NormalTok{(gdpPercap, lifeExp, }\AttributeTok{size =}\NormalTok{ pop, }\AttributeTok{color=}\NormalTok{continent)) }\SpecialCharTok{+}
  \FunctionTok{geom\_point}\NormalTok{() }\SpecialCharTok{+}
  \FunctionTok{theme\_bw}\NormalTok{()}

\FunctionTok{ggplotly}\NormalTok{(p)}
\end{Highlighting}
\end{Shaded}

\subsubsection{Figure}

\begin{Shaded}
\begin{Highlighting}[]
\FunctionTok{library}\NormalTok{(ggplot2)}
\FunctionTok{library}\NormalTok{(plotly)}
\end{Highlighting}
\end{Shaded}

\begin{verbatim}
Warning: package 'plotly' was built under R version 4.3.2
\end{verbatim}

\begin{verbatim}

Attaching package: 'plotly'
\end{verbatim}

\begin{verbatim}
The following object is masked from 'package:ggplot2':

    last_plot
\end{verbatim}

\begin{verbatim}
The following object is masked from 'package:stats':

    filter
\end{verbatim}

\begin{verbatim}
The following object is masked from 'package:graphics':

    layout
\end{verbatim}

\begin{Shaded}
\begin{Highlighting}[]
\FunctionTok{library}\NormalTok{(gapminder)}
\end{Highlighting}
\end{Shaded}

\begin{verbatim}
Warning: package 'gapminder' was built under R version 4.3.2
\end{verbatim}

\begin{Shaded}
\begin{Highlighting}[]
\FunctionTok{library}\NormalTok{(htmlwidgets)}
\end{Highlighting}
\end{Shaded}

\begin{verbatim}
Warning: package 'htmlwidgets' was built under R version 4.3.2
\end{verbatim}

\begin{Shaded}
\begin{Highlighting}[]
\FunctionTok{ggplotly}\NormalTok{(gapminder }\SpecialCharTok{\%\textgreater{}\%}
  \FunctionTok{filter}\NormalTok{(year}\SpecialCharTok{==}\DecValTok{1977}\NormalTok{) }\SpecialCharTok{\%\textgreater{}\%}
  \FunctionTok{ggplot}\NormalTok{( }\FunctionTok{aes}\NormalTok{(gdpPercap, lifeExp, }\AttributeTok{size =}\NormalTok{ pop, }\AttributeTok{color=}\NormalTok{continent)) }\SpecialCharTok{+}
  \FunctionTok{geom\_point}\NormalTok{() }\SpecialCharTok{+}
  \FunctionTok{theme\_bw}\NormalTok{())}
\end{Highlighting}
\end{Shaded}

\href{https://r-graph-gallery.com/interactive-charts.html}{r-graph-gallery}

\hypertarget{conclusion}{%
\subsection{Conclusion}\label{conclusion}}

\begin{itemize}
\tightlist
\item
  RStudio v2022.07 and later includes support for editing and preview of
  Quarto documents
\item
  Can use multiple languages in one document (think of
  \texttt{\{reticulate\}} for R \faIcon{left-right} Python)
\item
  Can use one document for multiple outputs (html, pdf, word, ppt)
\item
  Quarto extensions makes it easy to create document templates that can
  be used across platforms
\end{itemize}

\hypertarget{section-1}{%
\subsection{}\label{section-1}}

\includegraphics{TIOT_Intro_Quarto_files/mediabag/mdvdthkyou.gif}

\hypertarget{resources}{%
\subsection{Resources}\label{resources}}

\begin{itemize}
\tightlist
\item
  \href{https://quarto.org/docs/presentations/revealjs/demo/\#/title-slide}{Quarto
  Intro Slides}
\item
  \href{https://quarto.org/}{Quarto}
\item
  \href{https://rstudio-conf-2022.github.io/rmd-to-quarto/}{rstudio::conf
  2022 Workshop}
\item
  \href{https://emilhvitfeldt.com/blog\#category=quarto}{Emil Hvitfeldt
  Quarto blogs}
\item
  \href{https://github.com/mcanouil/awesome-quarto}{GitHub repository
  with most up-to-date Quarto resources}
\item
  \href{https://colorado.posit.co/rsc/tay-swift-tour/?mkt_tok=NzA5LU5YTi03MDYAAAGPv_qGZn8x95Dws5UJ3158TAN5EVLU68SYoumhNnRUtS8Jqw0Rnfx2k9OEXiNcWsQ1EiSEWvNctU3jPiOITrnXrkzgnRM0aSz1d3_5SP2e_W4}{Taylor
  Swift's Towering Year (made by Posit)}
\end{itemize}



\end{document}
